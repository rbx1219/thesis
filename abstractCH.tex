\begin{abstractCH}

\setlength{\baselineskip}{1.5em}
隨著科技以及知識的快速發展,世界上有許多事物開始仰賴最佳化這項技術。舉凡工程學上、經濟學上、甚或是計算機科學,都免不了最佳化這項過程。
最佳化這項技術困難的點在於我們常常無法得知問題的架構,也就是所謂的黑盒子最佳化。這時候我們能取得資訊的只有輸入以及輸出之間的對應關係而已,不知道問題的詳細組成讓最佳化無法以數學的模式進行。
在處理實數的最佳化問題上,取得探索以及開發之間的平衡是很重要的,一些傳統的方法,如實數編碼之延伸式精簡基因演算法,或是共變異數矩陣演化策略,在表現上受限於探索不足的問題。
在這篇論文中,我們提出了一個兩層的共變異數矩陣演化策略,試圖藉由演化區域最佳解,來尋找更加適應的解,並且將我們的系統架構映射到多臂吃角子老虎技術上,希望能更有效率的解開隱含在區域最佳解的資訊。
\\

\noindent
\textbf{關鍵詞} \\
\noindent
實數最佳化問題,黑盒子最佳化,共變異數矩陣演化策略,多臂吃角子老虎技術。
\end{abstractCH}
