\chapter{Experiment}
\label{ch:experiment}

In this chapter, we demonstrate the capability of our system by applying
the proposed algorithm into a set of test problems. In Section~\ref{sec:test
problem}, we introduce a set of test problems for real-valued function
optimization. The settings for the parameters and the results are shown
in Section~\ref{sec:experiment}. At last, we discuss about the phenomena
obtained from the results in Section~\ref{sec:discussion}.


\section{25 benchmark problems}
\label{sec:test problem}
To measure the performance comparing to other discretization methods, a set of
test function is demanded. As discussed in~\cite{liang2005novel}%J. J. Liang, P. N. Suganthan, and
%K.Deb, “Novel composition test functionsfor numerical global optimization,” in
%Proc. IEEE Swarm Intell.Symp., Pasadena, CA, Jun. 2005, pp. 68–75.
, many benchmark functions commonly suffer from several properties. We indicate
two important ones of them and give an instance to illustrate. First, global
optimum usually lies in the center of the search space. This property is taken
advantage by the design of algorithms with strong assumption to explore the
center of the domain. For example, the global
optimum of the test function given below are just in the center of search space. 
\[f(\mathbf{x}) = \sum\limits_{i=1}^d{x_i}^2,\]
where $\mathbf{x}$ is a vector composed of $(x_1, x_2,\ldots, x_d)$, $d$ is the
number of dimensions and $x_i \in (-5.12,5.12),\forall i \in \{1,2,\ldots,d\}$.

Second, local optima are
put along coordinate axes. In other words, there are no linkage between the
variables, hence the benchmark function is lack of the examination for  the ability to extract
linkage information between variables. For example, the formulation given below
consists of no dependency and global optimum can be easily extracted from local
optima
\[f(\mathbf{x}) = -\sum\limits_{i=1}^n{x_i},\]
where $\mathbf{x}$ is a vector composed of $(x_1, x_2,\ldots, x_d)$, $d$ is the
number of dimensions and $x_i \in (-5.12,5.12),\forall i \in \{1,2,\ldots,d\}$.

As a result, the design of the benchmark problems should at least consists of two
abilities.
\begin{itemize}
	\item The ability to shift\\
		This property gives an offset $\mathbf{o} = \mathbf{o_{new}} -
		\mathbf{o_{old}}$, where $\mathbf{o_{new}}$ is the new position of global optimum
		and $\mathbf{o_{old}}$ is the old one. In practice, the function
		evaluation is transformed from $f(\mathbf{x})$ to $F(\mathbf{x}) =
		f(\mathbf{x} - \mathbf{o})$. Note that the global optimum should not be
		shifted to the outside of the search space.
	\item The ability to rotate\\
		This property is introduced with an orthogonal matrix $\mathbf{M}$. The function
		evaluation is transformed to $F(\mathbf{x}) = f(\mathbf{Mx})$. Under the
		transformation, the local optima no longer put along the axes.
\end{itemize}

The benchmark problems proposed in CEC 2005 (Suganthan \textit{et al}.,
2005)~\cite{suganthan2005problem}
provides a set of problems without the disadvantages illustrated above.
The 25 benchmark problems are listed below where $D$ is the dimension,
$z$ is the shifted and sometimes rotated vector, $o$ is the offset and
$c_1,c_2,\ldots,c_{25}$ is the constant bias values for the 25 problems
so that it will be hard to guess the minimum.
$M$ is random orthogonal matrix to ratate the inputs.

\[F_1(x): \sum\limits_{i=1}^D {z_i}^2 + c_1.\]
\[F_2(x): \sum\limits_{i=1}^D\left(\sum\limits_{j=1}^i z_j\right)^2 +
c_2.\]
\[F_3(x): \sum\limits_{i=1}^D\left(10^6\right)^{\frac{i-1}{D-1}} {z_i}^2
+ c_3.\]
\[F_4(x): \left( \sum\limits_{i=1}^D\left(\sum\limits_{j=1}^i z_j\right)^2\right)
\times \left(1+0.4|N\left(0,1\right)| \right) + c_4.\]

\[F_5(x): max \{ {A_i}x-B_i \} + f\_bias_5 \quad i = 1,2,\ldots,D
\quad\text{for given vectors $A$, $B$ }\]

\[F_6(x): \sum\limits_{i=1}^{D-1} \left(
100\left({z_i}^2-{z_{i+1}}^2\right) + \left(z_i -1 \right) ^2 \right) +
c_6.\]

\[F_7(x): \sum\limits_{i=1}^D\frac{{z_i}^2}{4000} -
\prod\limits_{i=1}^Dcos\left(\frac{z_i}{\sqrt{i}}\right) +1 + c_7.\]

\[F_8(x):
  -20exp\left(-0.2\sqrt{\frac{1}{D}\sum\limits_{i=1}^D{z_i}^2}\right) -
exp\left(\frac{1}{D}\sum\limits_{i=1}^Dcos\left(2\pi z_i\right)\right)
+20 +e + c_8.\]


\[F_9(x): \sum\limits_{i=1}^D\left({z_i}^2 -10 cos\left(2\pi z_i
\right)+10 \right)+c_9.\]

\[F_{10}(x):\quad\text{Identical to $F_9$ except $z$ is rotated by
$M$ and $c_9$ is replaced by $c_{10}$.}\]

\[F_{11}(x):
  \sum\limits_{i=1}^D\left(\sum\limits_{k=0}^{k_{max}}\left(a^kcos\left(2\pi
  b^k\left(z_i + 0.5\right)\right)\right)\right) -
  D\left(\sum\limits_{k=0}^{k_{max}}\left(a^kcos\left(2\pi b^k \cdot
0.5\right)\right)\right) + c_{11},\]\quad\text\\ where $a = 0.5$, $b = 3$,
$k_{max}=20.$

\[F_{12}(x):\sum\limits_{i=1}^D\left(A_i-B_i\left(x\right)\right)^2+c_{12},\]
where $A_i =
\sum\limits_{j=1}^D\left(a_{ij}sin\alpha_j + b_{ij}cos\alpha_j\right)$,
$B_i(x) = \sum\limits_{j=1}^D\left(a_{ij}sin(x_j) +
b_{ij}cos(x_j)\right),$\\
$A$, $B$ are two $D$ square matrix,\\ $a_{ij}$, $b_{ij}$ are integer
random numbers in the range $[-100,100]$,\\ $\alpha_j$ are random numbers
in the range $[-\pi, \pi].$

\[F_{13}(x): \sum\limits_{i=1}^Df_7\left(f_2(z_{\left((i-1)\bmod D
+1\right)},
z_{\left((i-1)\bmod D +2\right)})\right)+c_{13},\]
where $f_7 = F_7 -c_8$, $f_2 = F_2-c_2,$ $z = x - o + \mathbf{1}.$

\[F_{14}(x):  \sum\limits_{i=1}^Df_{14}(z_{\left((i-1)\bmod D
+1\right)},
z_{\left((i-1)\bmod D +2\right)}),\]
where $f_{14}(x,y) = 0.5 + \frac{sin^2\left(\sqrt{x^2+y^2}
\right)-0.5}{\left(1+0.001(x^2+y^2)\right)^2},$ 
$z$ is rotated by $M.$
\\
$F_{15}$--$F_{25}$ are hybrid functions.
Giving $n$ basic functions used to construct the composition function, the
result should be 
\[F_j(x) = \sum\limits_{i=1}^n\left(w_i \times
\left(f_{i}^{'}((x-o_i)/\lambda_i * M_i) + c_i\right)\right) + c_j,\]
where $w_i$ is calculated below:\\$w_i = exp \left(- \frac{\sum\limits_{k=1}^D\left(x_k -
o_{ik}\right)^2}{2D\sigma_i^2} \right),$\\
$w_i = \left\{ 
    \begin{array}{l l}
          w_i & \quad \text{if $w_i = max(w_i)$}\\
          w_i*(1-max(w_i)^{10}) & \quad \text{otherwise}
                \end{array} \right. ,$
then normalize $w_i = w_i / \sum\limits_{i=1}^nw_i $

\[F_{15}(x): n = 10 ,\quad f_{1-2} = F_9,\quad f_{3-4}= F_{11},\quad f_{5-6} =
  F_7,\quad
f_{7-8} = F_8,\quad f_{9-10}= F_1, \]
where $\sigma_i = 1$,\\$\lambda =
[1,1,10,10,5/60,5/60,5/32,5/32,5/100,5/100]$,\\$M_i = I$. 

\[F_{16}(x):\text{ identical to $F_{15}$ except $M_i$ are different
linear transformation matrices.}\]

\[F_{17}(x): (F_{16}(x) - c_{16}) * (1+0.2|N(0,1)|) + c_{17}.\]

\[F_{18}(x): n = 10,\quad f_{1-2} = F_8, \quad f_{3-4} = F_9, \quad
f_{5-6} = F_1, \quad f_{7-8} = F_{11}, \quad f_{9-10}= F_7,\]
where $\sigma = [1,2,1.5,1.5,1,1,1.5,1.5,2,2]$,\\$\lambda =
[10/32,5/32,2,1,10/100,5/100,20,10,10/60,5/60]$,\\$M_i$ are all rotation
matrices. 

\[F_{19}(x): \text{identical to $F_{18}$}\]
except $\sigma = [0.1,2,1.5,1.5,1,1,1.5,1.5,2,2]$,\\$\lambda =
[0.5/32,5/32,2,1,10/100,5/100,20,10,10/60,5/60].$

\[F_{20}(x): \text{identical to $F_{18}$}\]
except $o_{1(2j)} = 5$, for $j = 1,2,...,\lfloor{D/2}\rfloor.$

\[F_{21}(x): n = 10,\quad f_{1-2} = F_{14}, \quad f_{3-4} = F_9, \quad
f_{5-6} = F_{13}, \quad f_{7-8} = F_{11}, \quad f_{9-10}= F_7,\]
where $\sigma = [1,1,1,1,1,2,2,2,2,2]$,\\$\lambda =
[25/100,5/100,5,1,5,1,50,10,25/200,5/200]$,\\$M_i$ are all orthogonal
matrices.

\[F_{22}(x): \text{identical to $F_{21}$}\]
except ${M_i}$'s condition numbers are
$[10,20,50,100,200,1000,2000,3000,4000,5000]$.

\[F_{23}(x): \text{identical to $F_{21}$}\]
except 
$x_j = y_j = \left\{ 
    \begin{array}{l l}
      x_j & |x_j -o_{1j} < 1/2|\\
      round(2x_j)/2 & \quad \text{otherwise}
                \end{array} \right.,$ for $j = 1,2,\ldots,D$.

\[F_{24}(x): n=10\]
$f_1 = F_{11},$\\
$f_2 = F_{14},$\\
$f_3 = F_{13},$\\
$f_4 = F_8   ,$\\
$f_5 = F_9   ,$\\
$f_6 = F_7   ,$\\
$f_7 = F{14}$ except that all $x_j$ which $|x_j| >=1/2$ should be
rounded,\\
$f_8 = F_9  $ except that all $x_j$ which $|x_j| >=1/2$ should be
rounded,\\
$f_9 = F_3,$\\
$f_{10}= F_1 \times (1+0.1|N(0,1)|).$\\
$\sigma_i$ = 2\\
$\lambda = [10,5/20,1,5/32,1,5/100,5/50,1,5/100,5/100],$\\
$M_i$ are all rotation matrices, condition numbers are
$[100,50,30,10,5,5,4,3,2,2].$

\[F_{25}(x): \text{identical to $F_{24}$}\]
except that no giving bound for the candidates.\\

The 25 functions given in the paper are categorized into several types, which are
\emph{Unimodal functions}, \emph{Basic multi-modal functions}, \emph{Expanded functions} and \emph{Hybrid composition functions}. Evaluation criteria are also introduced to measure the performance of algorithms. 
$F_1, F_2, \ldots, F_5$ are unimodal functions, and the first two are
easy to obtain high resolution solutions. $F_3$ and $F_5$ are easy to
move to the peaks with a strong local optimizer. $F_4$ is similar to
$F_2$ except that the former produces a perturbation by multiplying
a noise factor $(1+0.4N(0,1))$. 
$F_6$--$F_{14}$ are basic and expanded multi-modal problems. Among the 9
problems, $F_8$ is the so-called `Needle-In-A-Haystack' problem, which we can only obtain global optimum if the population is initialized around it. 
$F_{15}$--$F_{25}$ are composites of 10 basic functions, which are big challenges to search algorithms.
We can have an overview from Figure~\ref{fig:25testproblems}

\begin{figure}[htpb]
  \centering
  \includegraphics[bb= 43 16 783 584, clip, width=.7\textwidth]{benchmark.eps}
  \caption{Overview of the 4 categories}
  \label{fig:25testproblems}
\end{figure}

\section{Experiment settings and results}
\label{sec:experiment}

For describing abilities of generating better solutions, we set the number of
function evaluations to be $10^5$. 
That is to say, during each optimization we evaluate the function
exactly $10^5$ times. 
The best solution ever obtained is recorded after each run.
25 runs are executed for each problem, and the results are recorded and
sorted so that we can obtain the best, median and worst result of the 25
runs. 

For each problem, there is a table of fixed accuracy for each problem.
The accuracy is used to determine whether a candidate is close enough to
the global optimum.
Giving $\mathbf{x^\star}$ as the global optimum for function $i$, a solution
$\mathbf{x}$ is considered reaching the global optimum if $f_i(\mathbf{x}) -
f_i(\mathbf{x^\star}) < accuracy_i$. 
The table is given in Table~\ref{table:accuracy}.

\begin{table}
\centering
\begin{tabular}{c|c}
Function & Accuracy \\\hline
1-5      &  $1E-6$  \\\hline
6-16     &  $1E-2$  \\\hline
17-25    &  $1E-1$  \\
\end{tabular}
\caption{Accuracy value for each test function}
\label{table:accuracy}
\end{table}

The design of our algorithms are 2-stage CMA-ES and MAB-based CMA-ES.
The first aims to examine the advantage by clustering individuals
into groups.
The second describes that based on the clustering, we develop a soft
discretization by adaptively, iteratively replace a group to explore finer region
for better convergence value.
Therefore, we would like to examine things which:
\begin{enumerate}
  \item{2-stage CMA-ES increases the diversity}
  \item{MAB-based CMA-ES further increases the ability of searching}
  \item{MAB-based CMA-ES performs comparable to other discretization
    method}
  \end{enumerate}
The method we examine the abilities is first applying the algorithm to
the set of benchmark problems. For each problem, Our system records
the best solution ever found in each run. After 25 runs, the error
values, $f(\mathbf{x}) -f(\mathbf{x^\star})$, of each record is then
calculated.
At last, the 25 error values are sorted, and the best value, median
value, worst value along with mean value are highlighted to represent
the ability of the algorithm executing on the problem. Note that a
better solution is suggested to be with a smaller error value.
Note that our system uses the double-precision floating-point format to
represent the error value, and we can obtain from the definition that a
double-precision variable shows up to 15 decimal digits of precision.
As a result, the entry given `1E-15' means the resolution of the error
value is beyond expression because it is too small.   


As a consequence, to check the existence of diversity incremental,
we provide two tables for error values. The first one is the evaluation
results of original CMA-ES while the second is the results of the pure
2-stage CMA-ES.
Table~\ref{table:errorCMAES} is the evaluation results of CMA-ES
according to the following settings that:
\begin{itemize}
  \item{$\lambda = 20$, $\mu = 10$}
  \item{initial step size $\sigma$ = 1}
  \item{random vector in the search space as initial $m$}
\end{itemize}  
\begin{table}[h]
\footnotesize
  \centering
\begin{tabular}{@{}cccccccc@{}}
\toprule
Function&1       &2       &3       &4       &5       &6       &7     \\ \midrule
Best    &1.00E-15&1.00E-15&1.00E-15&1.00E-15&1.00E-15&1.00E-15&1.00E-15 \\
Median  &1.00E-15&1.00E-15&1.00E-15&1.00E-15&1.00E-15&1.00E-15&1.00E-15\\
Worst   &1.00E-15&1.00E-15&1.00E-15&1.41E+05&3.64E-12&3.99E+00&1.72E-02\\
Mean    &1.00E-15&1.00E-15&1.00E-15&6.28E+03&4.37E-13&3.19E-01&5.81E-03\\\midrule
Function&8       &9       &10      &11      &12      &13      &14      \\\midrule
Best    &2.00E+01&1.99E+00&2.98E+00&1.00E-15&1.00E-15&6.00E-01&4.07E+00\\
Median  &2.00E+01&5.97E+00&6.96E+00&1.00E-15&2.09E+01&1.03E+00&4.75E+00 \\
Worst   &2.13E+01&1.59E+01&1.19E+01&1.50E+00&1.06E+04&2.11E+00&5.00E+00\\
Mean    &2.01E+01&6.96E+00&7.20E+00&3.42E-01&1.54E+03&1.09E+00&4.71E+00\\\midrule
Function&15      &16      &17      &18      &19      &20      &21\\\midrule
Best    &1.02E+02&8.73E+01&9.69E+01&3.00E+02&3.00E+02&3.00E+02&3.00E+02\\
Median  &3.00E+02&1.22E+02&8.00E+02&8.00E+02&8.00E+02&9.53E+02&8.00E+02\\
Worst   &9.00E+02&9.84E+02&1.02E+03&9.95E+02&1.02E+03&1.01E+03&1.24E+03\\
Mean    &3.61E+02&2.26E+02&3.20E+02&8.02E+02&5.51E+02&8.80E+02&8.78E+02\\\midrule
Function&22      &23      &24      &25&&&\\\midrule
Best    &7.35E+02&5.56E+02&2.00E+02&4.07E+02&&&\\
Median  &7.54E+02&1.09E+03&2.00E+02&4.09E+02&&&\\
Worst   &8.00E+02&1.28E+03&9.00E+02&5.00E+02&&&\\
Mean    &7.68E+02&1.04E+03&3.28E+02&4.12E+02&&&\\\bottomrule
\end{tabular}
\caption{Error values for CMA-ES on functions 1-25 with $10^5$ function evaluations}
\label{table:errorCMAES}
\end{table}

In comparison with the original CMA-ES, 2-stage CMA-ES adds extra
mechanism to increase diversity, and the results of evaluations is given
in Table~\ref{table:2layercmaes}.
The settings are as follows:
\begin{itemize}
  \item{for each CMA-ES to be executed, $\lambda = 20$,
      $\mu=10$, $\sigma$ initialize to $1$}
  \item{for each inner CMA-ES, set $t$ = 1000 as the generation of each
    group}
  \item{an efficiently large population size $n$ = 450}
  \end{itemize}

\begin{table}[h]
\footnotesize
  \centering
\begin{tabular}{@{}cccccccc@{}}
\toprule
Function&1       &2       &3       &4       &5       &6       &7     \\ \midrule
Best    &1.00E-15&1.00E-15&1.00E-15&1.00E-15&1.00E-15&1.00E-15&1.00E-15 \\
Median  &1.00E-15&1.00E-15&1.00E-15&1.00E-15&3.64E-12&1.00E-15&1.00E-15\\
Worst   &1.00E-15&1.00E-15&1.00E-15&2.82E+04&7.28E-12&3.99E+00&2.46E-02\\
Mean    &1.00E-15&1.00E-15&1.00E-15&4.06E+03&3.20E-12&1.59E-01&4.83E-03\\\midrule
Function&8       &9       &10      &11      &12      &13      &14      \\\midrule
Best    &2.00E+01&1.99E+00&1.99E+00&1.00E-15&1.00E-15&3.27E-01&3.86E+00\\
Median  &2.10E+01&5.97E+00&5.97E+00&1.00E-15&4.22E+02&9.30E-01&4.59E+00 \\
Worst   &2.14E+01&9.95E+00&1.39E+01&2.39E+00&1.06E+04&1.79E+00&4.99E+00\\
Mean    &2.08E+01&6.25E+00&6.65E+00&4.23E-01&3.14E+03&1.01E+00&4.56E+00\\\midrule
Function&15      &16      &17      &18      &19      &20      &21\\\midrule
Best    &9.87E+01&9.56E+01&1.03E+02&3.00E+02&3.00E+02&3.00E+02&3.00E+02\\
Median  &4.00E+02&1.11E+02&1.32E+02&9.40E+02&9.47E+02&9.42E+02&8.00E+02\\
Worst   &9.00E+02&9.00E+02&1.02E+03&1.01E+03&1.01E+03&9.99E+02&1.25E+03\\
Mean    &4.13E+02&2.51E+02&4.25E+02&8.57E+02&8.84E+02&8.45E+02&8.76E+02\\\midrule
Function&22      &23      &24      &25&&&\\\midrule
Best    &7.27E+02&5.54E+02&2.00E+02&3.95E+02&&&\\
Median  &7.49E+02&1.09E+03&2.00E+02&3.96E+02&&&\\
Worst   &8.02E+02&1.27E+03&9.00E+02&5.00E+02&&&\\
Mean    &7.64E+02&1.04E+03&3.48E+02&4.05E+02&&&\\\bottomrule
\end{tabular}
\caption{Error values for 2-stage CMA-ES on functions 1-25 with $10^5$ function evaluations}
\label{table:2layercmaes}
\end{table}

As a modification of 2-stage CMA-ES, MAB-based CMA-ES being a softer
discretization method takes advantage from maintaining an outer CMA-ES
to ceaselessly exploring a finer regions for better solutions.
For observing implicit information among groups, the modified algorithm
keeps a slower convergence in order to make difference with the previous
approach, which observe only implicit information among local optima.
The settings are as follows:
\begin{itemize}
  \item{for each CMA-ES to be executed, $\lambda = 20$,
      $\mu=10$, $\sigma$ initialize to $1$}
  \item{for each inner CMA-ES, set $t$ = 30 as the generation of each
    group}
  \item{an efficiently large population size $n$ = 450}
\end{itemize}


\begin{table}[h]
\footnotesize
  \centering
\begin{tabular}{@{}cccccccc@{}}
\toprule
Function&1       &2       &3       &4       &5       &6       &7     \\ \midrule
Best    &1.00E-15&1.00E-15&1.00E-15&1.00E-15&1.00E-15&1.00E-15&1.00E-15 \\
Median  &1.00E-15&1.00E-15&1.00E-15&1.00E-15&1.00E-15&1.00E-15&1.00E-15\\
Worst   &1.00E-15&1.00E-15&1.00E-15&3.18E-06&1.00E-15&1.19E-03&1.48E-02\\
Mean    &1.00E-15&1.00E-15&1.00E-15&2.65E-05&1.00E-15&4.78E-05&1.28E-03\\\midrule
Function&8       &9       &10      &11      &12      &13      &14      \\\midrule
Best    &2.00E+01&1.99E+00&1.99E+00&1.00E-15&1.00E-15&3.39E-01&3.43E+00\\
Median  &2.00E+01&3.98E+00&3.98E+00&1.00E-15&1.00E-15&5.79E-01&4.27E+00 \\
Worst   &2.04E+01&9.95E+00&7.96E+00&1.07E+00&1.56E+03&4.56E+00&4.56E+00\\
Mean    &2.01E+01&4.74E+00&4.18E+00&8.53E-02&1.21E+02&6.71E-01&4.14E+00\\\midrule
Function&15      &16      &17      &18      &19      &20      &21\\\midrule
Best    &7.72E+01&5.61E+01&4.23E+01&3.00E+02&3.00E+02&3.00E+02&3.00E+02\\
Median  &1.14E+02&9.86E+01&1.01E+02&3.00E+02&5.00E+02&3.00E+02&5.00E+02\\
Worst   &4.00E+02&1.34E+02&1.23E+02&9.50E+02&9.84E+02&9.58E+02&8.00E+02\\
Mean    &1.35E+02&9.52E+01&9.80E+01&5.20E+02&5.51E+02&5.15E+02&5.27E+02\\\midrule
Function&22      &23      &24      &25&&&\\\midrule
Best    &7.30E+02&5.54E+02&2.00E+02&4.03E+02&&&\\
Median  &7.67E+02&5.59E+02&2.00E+02&4.04E+02&&&\\
Worst   &8.00E+02&1.19E+03&8.00E+02&4.05E+02&&&\\
Mean    &7.59E+02&6.75E+02&2.36E+02&4.04E+02&&&\\\bottomrule
\end{tabular}
\caption{Error values for MAB-based CMA-ES on functions 1-25 with $10^5$ function evaluations}
\label{table:errorMABcmaes}
\end{table}

To measure the performance of MAB-based CMA-ES, we introduced a
well-known discretization method, SoD,  and compare the evaluation
results. The competition criteria can be the resolution of best results,
the number of solved functions, the median values, etc. 
Therefore, the parameter setting of SoD  is given below:
\begin{itemize}
  \item{the population size = 250}
  \item{the probability of crossover is 0.975}
  \item{the tournament size = 8}
  \item{the initial density $\gamma$ = 0.5}
  \item{the decadence factor $\epsilon$ = 0.998}
  \item{for every 5 generations a local optimizer is adapted}
  \end{itemize}


\begin{table}[h]
\footnotesize
  \centering
\begin{tabular}{@{}cccccccc@{}}
\toprule
Function&1       &2       &3       &4       &5       &6       &7     \\ \midrule
Best    &5.68E-14&5.68E-14&1.14E-13&3.45E-08&1.06E-04&1.71E-13&9.38E-13 \\
Median  &6.82E-13&1.02E-12&1.25E-12&2.33E-04&1.33E-03&1.24E-11&1.03E-01\\
Worst   &4.43E-12&1.68E-11&1.93E-04&2.58E-02&1.54E+00&5.54E+01&5.34E-01\\
Mean    &1.29E-12&2.11E-12&7.72E-06&2.13E-03&8.77E-02&5.77E+00&1.42E-01\\\midrule
Function&8       &9       &10      &11      &12      &13      &14      \\\midrule
Best    &2.00E+01&1.71E-13&2.98E+00&5.23E-01&5.68E-14&9.91E-03&2.16E+00\\
Median  &2.00E+01&6.82E-13&7.96E+00&2.21E+00&1.76E-12&3.85E-01&3.06E+00 \\
Worst   &2.00E+01&8.01E-12&1.39E+01&5.53E+00&1.56E+03&8.54E-01&4.00E+00\\
Mean    &2.00E+01&1.33E-12&7.84E+00&2.35E+00&6.39E+01&4.37E-01&2.97E+00\\\midrule
Function&15      &16      &17      &18      &19      &20      &21\\\midrule
Best    &2.84E-14&9.43E+01&1.13E+02&3.00E+02&3.00E+02&3.00E+02&3.00E+02\\
Median  &1.72E-12&1.10E+02&1.25E+02&8.00E+02&8.00E+02&8.00E+02&8.00E+02\\
Worst   &4.24E+02&1.30E+02&1.73E+02&9.51E+02&9.68E+02&9.28E+02&1.13E+03\\
Mean    &1.55E+02&1.11E+02&1.30E+02&7.28E+02&6.61E+02&5.86E+02&6.87E+02\\\midrule
Function&22      &23      &24      &25&&&\\\midrule
Best    &7.27E+02&5.59E+02&2.00E+02&4.21E+02&&&\\
Median  &7.40E+02&9.71E+02&2.00E+02&4.40E+02&&&\\
Worst   &8. 0E+02&1.24E+03&2.00E+02&1.48E+03&&&\\
Mean    &7.47E+02&8.14E+02&2.00E+02&6.35E+02&&&\\\bottomrule
\end{tabular}
\caption{Error values for SoD on functions 1-25 with $10^5$ function evaluations}
\label{table:errorSoD}
\end{table}

\section{Discussions}
  \label{sec:discussion}

In this section, we further investigate the results and summarize
accordingly.
To give a brief overview firstly, we give a table for recognizing solved
function for the proposed algorithms along with original CMA-ES and
rECGA with SoD.
According to the category given by the benchmark, there are 3 columns
for unimodal problems, basic multi-modal problems and other difficult
problems respectively.

\begin{table}
\centering  
\begin{tabular}{cccc}
Algorithm & Unimodal & Basic multi-modal & Expanded and Hybrid \\\hline  
CMA-ES   & 1,2,3,4,5 & 6,7,11,12 & -- \\
2-stage CMA-ES & 1,2,3,4,5 & 6,7,11,12 & -- \\
MAB-based CMA-ES & 1,2,3,4,5 & 6,7,11,12 & -- \\
rECGA with SoD & 1,2,3,4 & 6,7,9,12 &13,15 \\\hline
\end{tabular}
  \caption{The index of solved functions for the algorithms}
  \label{table:solved}

\end{table}

Table~\ref{table:solved} shows that in unimodal problems, the CMA-ES series algorithms
are able to reach the global optimum.
However, $F_5$ is difficult for rECGA to solve due to the property of
the function that it is easy to solve if the initial population are
located at bounds, which rECGA fails to solve due to SoD's feature.
Besides, CMA-ES series and rECGA solve 4 functions among the 7 basic
multi-modal problems, while 11 is solvable for the former and 9 for the
later.
At last, among 14 difficult problems, CMA-ES series fail to reach
suitable resolution to conquer the problems, while rECGA success to
tackle $F_{13}$ and $F_{15}$.

First, the comparison between table~\ref{table:errorCMAES} and table
~\ref{table:2layercmaes} is investigated in the following.
Among unimodal problems, the best obtained error values are all beyond
expression in a 64 bits variable. 2-stage CMA-ES performs slightly worse
than original CMA-ES on median value of $F_5$.
We surmise this situation is caused by the insufficient generations for
each group, but this is still a comparable result with respect to other
evolutionary algorithms.
In the remaining multi-modal problems, 2-stage CMA-ES outperforms the original
CMA-ES in best error values on 8 problems while the original
wins in 2 problems.
This confirms our hypothesis that maintaining large population helps to
increase diversity along with the ability of exploration, and better
solutions are evolved more efficiently.

Second, we propose a modification on 2-stage CMA-ES, which is referred
to as MAB-based CMA-ES and able to extract implicit information to
adjust the distribution of population.
By comparing table~\ref{table:2layercmaes} and
table~\ref{table:errorMABcmaes}, the problem of 2-stage CMA-ES which
suffers from insufficient generations is eliminated on $F_5$.
Besides, when taking the best values into consider in the 20 multi-modal
problems, the 2-stage CMA-ES wins in three problems while the MAB-based
CMA-ES wins in four.
To further investigate the best error values, the result indicates that
the modified approach does not outperform the 2-stage CMA-ES in the
measurement of exploration.
However, by considering the median values, the MAB-based CMA-ES
significantly outperforms the 2-stage CMA-ES in 15 problems, while on the
other hand the 2-stage CMA-ES wins only in 2 problems.
This result inspires us that MAB-based CMA-ES is able to maintain the
ability of keeping diversity in 2-stage CMA-ES and meanwhile provide
stability.
The induction can be enhanced by observing the mean error value, the
MAB-based CMA-ES loses in none of the problems.

Finally, the MAB-based CMA-ES is compared with rECGA with SoD.
The collection of error values are shown in
table~\ref{table:errorMABcmaes} and table~\ref{table:errorSoD}
respectively.
SoD being an adaptively discretization method improves rECGA
significantly comparing to traditional methods,FHH and FWH.
Since we aim to propose a more elegant discretization method by
considering the implicit information between groups, it is unavoidable
to put them together for comparison.
Though it has been shown that MAB-based CMA-ES can solve 9 problems while rECGA
with SoD is able to solve 10 problems, each of them does not dominate
the other in the solved problem.

To further compare them, we can first analyze unimodal problems.
MAB-based CMA-ES being an extension of CMA-ES keeps the well-known ability of local
optimization, which usually reaches the global optimum in a very high resolution.
Thus the unimodal problems are considered easy for MAB-based CMA-ES, and
the results shows that MAB-based CMA-ES can solve $F_1,F_2,F_3$ and
$F_5$ at all time while more than half for function 4.
On the other hand, SoD also solves $F_1$ and $F_2$ at all time and gives
comparable results for $F_3$ and $F_4$.
As we have shown, SoD encounters difficulty in $F_5$.

Among basic multi-modal problems, it is vague to distinguish which
performs better when observing the best values.
As described above, both algorithms do not dominate the other in the
series of test problem.
However, we can obtain from the 2 tables that MAB-based CMA-ES
introduces better median values except $F_9$. 
Five of them are smaller even in different magnitude.

Finally in expanded and hybrid problems, our algorithm fails to reach
the global optima in $F_13$ and $F_15$, in which rECGA with SoD obtains
high resolution results.
Among the 13 problems, MAB-based wins in four problems with respect to
the best values, while rECGA with SoD wins four problems, too.
Again we lay emphasis on the median values, CMA-ES wins in 8 problems
while the other one wins in 4 problems.
According to a better performance of median values in basic multi-modal
problems and difficult problems, our induction is enhanced that
MAB-based CMA-ES offers a stabler approach for real-valued function
optimization.

Referring to Table~\ref{table:best}, Table~\ref{table:median} and
Table~\ref{table:worst}, the robustness of our algorithm compared to
the original algorithms are shown.
This provides an evidence that our algorithm improves the ability of
exploration and maintains the ability of exploitation so that we are
able to defeat rECGA with SoD in Unimodal problems.

Also, the disadvantage of our approach takes place when facing simple
problems, such as $F_1,F_2,\ldots,F_7$. Because of maintaining
multi arms simultaneously, we should have keep track of each arm at all
time.
But in simple problems, each group are driven towards the same
direction.
This means that we spends much extra resource for tracing multi arms.
The evidence are shown in Table~\ref{table:tradeoff}, it is obvious that
our algorithm spends more $N_{f_e}$ comparing to the original CMA-ES.

\begin{table}[h]
\centering
\begin{tabular}{|c|c|c|}

\hline
      & \begin{tabular}[c]{@{}c@{}}Original\\ CMA-ES\end{tabular} &
      \begin{tabular}[c]{@{}c@{}}MAB-based\\ CMA-ES\end{tabular} \\ \hline
      $F_1$ & 3398.28 & 3721.76 \\ \hline
      $F_2$ & 4107.76 & 20836.84\\ \hline
      $F_3$ & 8047.36 & 24852.80\\ \hline
      $F_4$ & 5921.76 & 19489.33\\ \hline
      $F_5$ & 9268.00 & 25395.72\\ \hline
      $F_6$ & 11323.48& 37854.22\\ \hline
      $F_7$ & 4612.00 & 14052.60\\ \hline
\end{tabular}
\caption{The average $N_{f_e}$ for successful evaluations of functions 1--7 }
\label{table:tradeoff}
\end{table}


\begin{table}[h]
\centering
  \begin{tabular}{|c|c|c|}
\hline
      & \begin{tabular}[c]{@{}c@{}}Original\\ CMA-ES\end{tabular} &
      \begin{tabular}[c]{@{}c@{}}MAB-based\\ CMA-ES\end{tabular} \\ \hline
U     & --                                                       & --              \\ \hline
M     & --                                                       & 1               \\ \hline
E     & --                                                       & 2               \\ \hline
H     & --                                                       & 6               \\ \hline
Total & --                                                       & 9               \\ \hline
\end{tabular}
\begin{tabular}{|c|c|c|}
\hline
      & \begin{tabular}[c]{@{}c@{}}rECGA+\\ SoD\end{tabular} & \begin{tabular}[c]{@{}c@{}}MAB-basd\\ CMA-ES\end{tabular} \\ \hline
U     & --
& 5                                                    \\ \hline
M     & 1                                                         &
5                                                    \\ \hline
E     & 2                                                         &
--                                                    \\ \hline
H     & 4                                                         & 4                                                    \\ \hline
Total & 7                                                        &
14                                                   \\ \hline
\end{tabular}
\caption{Comparisons of best run}
\label{table:best}
\end{table}
      
\begin{table}[h]
  \centering
\begin{tabular}{|c|c|c|}
\hline
      & \begin{tabular}[c]{@{}c@{}}Original\\ CMA-ES\end{tabular} & \begin{tabular}[c]{@{}c@{}}MAB-basd\\ CMA-ES\end{tabular} \\ \hline
U     & --
& --                                                         \\ \hline
M     & --
& 3                                                         \\ \hline
E     & --                                                       & 2                                                         \\ \hline
H     & 1
& 9                                                         \\ \hline
Total & 1                                                        &
14                                                        \\ \hline
\end{tabular}
\begin{tabular}{|c|c|c|}
\hline
      & \begin{tabular}[c]{@{}c@{}}rECGA+\\ SoD\end{tabular} & \begin{tabular}[c]{@{}c@{}}MAB-basd\\ CMA-ES\end{tabular} \\ \hline
U     & --
& 5                                                    \\ \hline
M     & 1
& 5                                                    \\ \hline
E     & 2
& --                                                    \\ \hline
H     & 2
& 8                                                    \\ \hline
Total & 5
& 18                                                    \\ \hline
\end{tabular}
\caption{Comparisons of median run}
\label{table:median}
\end{table}


\begin{table}[h]
  \centering
\begin{tabular}{|c|c|c|}
\hline
      & \begin{tabular}[c]{@{}c@{}}Original\\ CMA-ES\end{tabular} & \begin{tabular}[c]{@{}c@{}}MAB-basd\\ CMA-ES\end{tabular} \\ \hline
U     &1     &1                                                         \\ \hline
M     &--    &7                                                         \\ \hline
E     &2     &--                                                        \\ \hline
H     &--    &10                                                        \\ \hline
Total &3     &18                                                        \\ \hline
\end{tabular}
\begin{tabular}{|c|c|c|}
\hline
      & \begin{tabular}[c]{@{}c@{}}rECGA+\\ SoD\end{tabular} & \begin{tabular}[c]{@{}c@{}}MAB-basd\\ CMA-ES\end{tabular} \\ \hline
U     &-- &5                                                    \\ \hline
M     & 1 &5                                                    \\ \hline
E     & 1 &1                                                    \\ \hline
H     & 3 &7                                                    \\ \hline
Total & 5
& 18                                                    \\ \hline
\end{tabular}
\caption{Comparisons of worst run}
\label{table:worst}
\end{table}

To sum up, we propose several experiments to show that (1) clustering is
helpful to increase the diversity of the population comparing to pure
CMA-ES, (2) the MAB-based CMA-ES offers a stable approach and maintains
the ability of exploration enhanced by clustering simultaneously and
(3) MAB-based CMA-ES provides a comparable result with the adaptive
discretization method, SoD.
In the best-error-value competition with SoD, MAB-based CMA-ES obtains
smaller values on 14 problems, while rECGA with SoD wins on 4
problems.
Further in median-error-value competition, MAB-based CMA-ES obtains
smaller values on 18 problems, while the opposite obtains on 5 problems.
The result provides the evidence that MAB-based CMA-ES is a technique
that:
\begin{enumerate}
\item{Generates high resolution solutions once the global optimum is
  roughly located}
\item{Performs stably and compratably to other evolutionary algorithms}
  \end{enumerate}

