\begin{acknowledgementsCH}

\setlength{\baselineskip}{1.5em}
這篇論文的完成,我最感謝的是于天立老師。作為一個指導教授,于老師具備專業的學術涵養來應付學生的各種難題;作為一個老師,關心學生之餘也從不會給予額外的壓力,讓有熱情的學生能夠放手去做研究。對事不對人的人生觀也幫助我培養不以言廢人的風度;就算把老師當成朋友,老師的在新的科技以及電玩上的知識也不曾讓人在他身上感覺到年紀過。總的而言,站在巨人的肩膀上,即使我再怎麼不力爭上游,老師也能讓我的眼界比起比一般的研究生更加遼闊。

第二要感謝的是蘇倚恩學長以及林廷舟,這篇論文的完成,很大一部份仰賴於兩位在技術上以及寫作上給我的挫折以及建議。

接下來想要感謝實驗室的成員,李玄、歐發,張仁豪學長讓我知道有熱情有能力的研究生可以很享受研究的生活;陳韋名、周志遠、邵中昱以及許博竣陪我一起修課、一起打電動,讓我在實驗室不是孤單一個人。王士銘、許儲羽、吳潔薇、許世煥、童宇凡、吳家輝這幾位學弟妹能對我的研究實作上給了不少的建議。而呂昶毅、張家華、曾麒元三位學弟妹也很有耐心的幫助我度過了數次的rehearsal並且給我很大的幫助。

再來,我想謝謝我在資訊系籃的同伴,最感謝的一定是神魔之塔群組內的所有夥伴,不管是場上或是場下總是很關心我並且在我失意的時候鞭策以及鼓勵我。學弟妹,尤其是林善偉隱藏在酸言酸語的關心也很受用,b93學長鍾以千總是無償的陪我除蟲,b78莊學長特地來幫助我完成口試,都是無法忘記的恩情。

倒數第二要感謝的是我的父母以及弟妹,沒有他們在經濟上以及心靈上的支援,我沒有辦法完成這份學業。

最後我要謝謝我的未婚妻,饒卿玄,在我最後衝刺的時間總是不斷的鼓勵我,忍受我的壞脾氣,從吃飯喝水到婚禮的準備都由她一手包辦,因此我能心無旁騖的準備,也因此才有這篇論文的產生。

謹以此段文字感謝以上諸位,沒有你們就沒有今天以及未來的我。

\end{acknowledgementsCH}
