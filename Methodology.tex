\chapter{Methodology} \label{ch:method}

In previous chapter, the hypothesis of this work is provided.
This chapter describes how we implemented the system in detail.
Section~\ref{sec:overview}
gives a brief overview for the system.
A pure 2-layer CMA-ES will be described in~\ref{sec:2layer} and a
modified one will be given in~\ref{MAB-based}.

\section{Overview} \label{sec:overview}

According to our hypothesis, the proposed system must have the ability of
clustering.
This is because the initialization is assumed to be uniformly
sampled in the search space without any prerequisite, and we wish to divide the
population into groups under some criteria such as distance between solutions.
Besides, in order to extract information between groups, our system
demands a method to evolve the divided groups.
There is an objective function to obtain the values of candidate
solutions, but there is no measurement to evaluate fitness of the
groups.
A simple method is using a representative solution, such as the best
member or the central of the group and the member with median objective
function value.
At last, we have to define the timing of the switching time from the
original computation level to the second layer computation. 

A simple flow can be given in Algorithm~\ref{algo:overview}.

\begin{algorithm}[H]
  \label{algo:overview}
  Initializing population\;
	Clustering\;
  \While{not terminate}
  {
    \While{any group has not been evolved for certain generation}
    {
    Evolving each group\;
    }
    Selecting candidates for each group\;
    Evolving the selected solutions\;
  }
  \caption{Overview of the system}
\end{algorithm}
In the following sections, related works about
division and the method of evaluating second layer will be introduced.
Section~\ref{sec:2layer}
gives a simple instance for the system and section~\ref{MAB-based} gives a
modified one.

\section{2-Layer CMA-ES} \label{sec:2layer}

To verify the assumption that dividing the population increases the
diversity, we designed an algorithm in this section. 
%To give instances for the template, we are going to supply implementation for
%the division and the evaluation strategy for the 2-layers approach.
As the population are represented as vectors, we divided the population
according to space locality.
The reason to do so is that a group in a specific region would converge
to the same local optima.
In order to keep diversity, all of groups are suggested to maintain
their representativeness, which we can gain from space locality.
As a result, a clustering method able to gather nearby solutions
together is essential.

\subsection{k-means}

To split the population into groups according to space locality, the
coming idea is applying the technique of clustering.
Many approaches have been proposed in the field of clustering.
\emph{k-means}, a member of the family of vector quantization, is a
popular skill for cluster analysis in data mining.
\emph{k-means} aims to partition $n$ observations into $k$ clusters
where $k < n $.
Obtaining the best partition without given $k$ and dimension of
observations $d$ is proved to be computationally difficult (NP-hard).
As a result, to determine the value of $k$, this thesis takes the rule
of thumb which sets $k = \sqrt{\frac{n}{2}}$.

There are some efficient heuristic algorithms to make \emph{k-means}
converge to a local optimum quickly.
An iterative refinement approach is provided: \emph{k-means} selects
out $k$ centroids as the center of $k$ groups.
Subsequently, every observation are assigned to the center which is closest.
Before the end of an iteration, the center is updated by the mean of all vectors
belong to it.A detail formulation can be given as following.

Given a set of observations $\left( x_1, x_2, \ldots, x_n\right)$ where each
observation is a $d$-dimensional vector, \emph{k-means} clustering aims to
partition the set into $k$ sets $\mathcal{S} = \{ S_1, S_2, \ldots, S_k \}$ such
that the sum of the squares within cluster is minimized.
That is to say, the objective is to find:
\[arg\min_{\mathcal{S}}\sum\limits_{i=1}^k \sum\limits_{x \in S_i}
\|x-\mu_i\|\], where $\mu_i$ is the arithmetic mean of vectors in $S_i$.

In this thesis the initialization method is referred to as \emph{Forgy method}.
The overall procedure of the modified \emph{k-means} can be algorithmically
outlined as Algorithm~\ref{algo:kmeans}.

\begin{algorithm}[H]
  \label{algo:kmeans}
	 \KwIn{$k$, $d$, $\{o_1, o_2,\ldots, o_n\}$ as
	observations} \KwOut{$\mathcal{S}$} Initial: $m_1, m_2,\ldots, m_k$ are
	random selected from observations as initial centers\; \While{At least one
	of the observations moves to other group} { \For{j = 1 to k} { $S_j =
		\emptyset$	\; } \For{i = 1 to n} { assign $o_i$ to $S_j$ if $o_i$ is
		closest to $m_j$ among the $k$ centers\; } \For{j = 1 to k} { $m_j$ is
	updated by the arithmetic mean of all vectors $\in S_j$\; } } 
  \caption{k-means}
\end{algorithm}

The described procedure shows an approach that can quickly obtain local
optimum.
Due to the initialization of $m$ centers, the proposed procedure is an
unstable approach.
In other words, with different initial set of centers, the output of the
algorithm is different even consisting of the same observations.
However, in practice, this is an method that is easy to
implement and performs acceptably.


\subsection{Methodology}

The characteristic of the algorithm is that it maintains a larger
population, which is inefficient in the original CMA-ES, so that it may
obtain better diversity of a real-valued problem.
Recalled from the flow giving in section~\ref{sec:overview}, 2 criteria
have not been determined yet: (1) the method to choose representative
solutions and (2) the timing to switch from bottom layer to the top one.
In this algorithm, we take the local optima of groups after executing
CMA-ES for a given generation as the representative solutions.
Also, a given generation is adopted to control the number of iterations
for each group.
The main purpose of this 2-layer CMA-ES is to distinguish the ability of
division, and roughly evolve the groups to extract the information
inside.
Consequently, the remaining criteria are set to be simple so that we can
verify our assumption directly.

A detail algorithm can be illustrated as following\\
\begin{algorithm}[H]{
		\caption{2-layer CMA-ES}
    \KwIn{$n$,$t$}
    \KwOut{best solution ever evolved}
    Uniformly sampled population of size $n$\;
    $k = \sqrt{\frac{n}{2}}$\;
    Integrating \emph{k-means} with \emph{Frogy-method} to cluster the
    $n$ individuals into $k$ groups\;

		$C\leftarrow$ array with size $k$\;
    \For{i = 1 to k}
    {
      Optimizing group$_i$ by adopting CMA-ES for $t$ generations\;
			$C_i\leftarrow$ best solution\;
      Applying CMA-ES to evolve the population consisting of local
      optima of groups as known as $C$ until terminated.\;
    }
}
\end{algorithm}

The proposed algorithm lays emphasis on maintaining the diversity, and
we can surmise that there should be trade-off to benefit from the large
population.
This is obvious in minimizing convex functions, which we refer to as
unimodal problems.
In a unimodal problem, we can gain from the algorithm that $C_1, C_2,
\ldots, C_k$ are driven toward the only optimum for CMA-ES being a good
local optimizer.
Accordingly, optimization is repeated for $k-1$ times, which is
expensive, especially when evaluating the objective function is costive.


\section{MAB-based CMA-ES} \label{MAB-based}

The previous section introduces a 2-layer CMA-ES trying to benefit from
a large population.
However, the algorithm does not take group priority into consideration,
and we wondered if any priority between groups should be claimed.
That is to say, if a group seems impossible for the global optimum to
exist, we should focus on the possible ones in order to obtain better
solutions.
Furthermore, we would like to design a mechanism to drop off groups that
contribute nothing anymore for extracting information between groups.

To sum up, this section introduces a modified version of the algorithm
proposed in the previous section.
The motivation behind this design is to make the search escape from the
ruggedness hazard and jump to a better cliff in the search space.
What we try to offer in this section is providing a selection and
dropping strategy in order to extract the information hided between
groups.

\subsection{Multi-armed bandit problems}

Multi-armed bandit (MAB) problems are a class of sequential resources
allocating problems.
MAB problems are about allocating resources among several competing
objects, which addresses the priority between competitors, and
we can model our problem finding regions where better solutions may
locate as a MAB problem.
We give detail introduction to the MAB problems and
integrate the method to modify the proposed 2-layer CMA-ES.

\subsubsection{Problem description}

We are going to illustrate the problem in advance by giving an
example.
Assume a gambler is with a row of $k$ slot machines which are
also referred to as `one-armed bandits', and he decides to play them in
a sequence.
The slot machines output rewards according to built-in distributions,
and the distributions are different between each machines.
Without any prior knowledge (e.g.\ distributions), what the gambler can
obtain is only the given reward every time he pays and pulls the bandit.
The goal of the problem in this example is to maximize the rewards
according to the recorded sequence of rewards along with the index of
machines he played.

In a formal expression, the slot machines stand for $k$ distributions
$D_1, D_2, \ldots, D_k$, which with $\mu_i,\mu_2,\ldots,\mu_k$ as
expectation values and variances $\sigma_1, \sigma_2,\ldots,\sigma_k$.
A random variable $X_{i,n}$ is the recorded sampled values in
distribution $D_i$ and $n$ stands for the time the machine has been
played.
For example, $X_{4,5}$ is the 5-th sampled value of $D_4$.
A popular performance measurement called \emph{expected regret} can be
defined with respect to a fixed turns $T$ that: \[R_T = T\mu^\star -
\sum\limits_{t=1}^T \mu_{j(t)},\] where $j(t)$ is a mapping function
that indicates the index of machine pulled in $t$-th round while
$\mu^\star$ is the best expectation reward among $k$ distributions.
Besides, the lower bound of the growing rate for $R_T$ is proved to be
$\mathcal{O}(\ln{n})$.
As consequence, the problem aims to find the optimal mapping $t
\rightarrow j(t)$ to minimize $R_T$, and a solution can be considered
appropriate for solving exploration-exploitation trade-off if the policy
makes the growth rate of \emph{expected regret} is within a constant
factor of the lower bound.

\subsubsection{Related Approach}

We have to design an adaptive sampling strategy, based on the previous
observed samples, to select an arm at each round.
We are going to introduce a solution named \emph{upper confidence bound}
(UCB) proposed by Auer et al. in 2002, and Auer in 2003.
UCB algorithms are widely integrated into MAB problems and considered an
more elegant and simpler implementation when it comes to the uncertainty
problems.
The main concept of the UCB family is to reach a balance between average
reward sampled in $(0,1)$ and the number of times a machine is played.
That is to say, for $D_1, D_2, \ldots, D_k$, the next distribution we
tend to sample from should be the one giving better average sample
reward, but distributions with worse reward are possibly to be chosen
once they become relatively unknown to the often sampled ones.
In other words, the way to design the algorithm is to give a measurement
combining average reward and the number of sampled times for each
distribution. 

Algorithm \emph{UCB1} is a simple way to combine the factors and is
proved to achieve logarithmic \emph{expected regret} over $n$ without
any preliminary knowledge about the distributions.
The combination can be formulated as sum of the two terms as following
when considering machine $j$: \[\bar{x_j} +
\sqrt{\frac{2\ln{n}}{n_j}},\] where $\bar{x_j}$ is the average of
rewards obtained in $D_j$ so far, $n$ is the total round the system has
sampled and $n_j$ is the total round for distribution $D_j$.
Thus, the optimal mapping is in the following form: \[j(t) = \arg\max_{i
\in \{1,2,\ldots,k\}}\bar{x_i} + \sqrt{\frac{2\ln{n}}{n_i}}.\]

Along with \emph{UCB1}, the author proposed another algorithm which adds
an extra element into \emph{UCB1}, and claimed better performance in
practice, although without any theoretical guarantee.
The newly proposed algorithm `\emph{UCB1-tuned}' is a modification of
\emph{UCB1} which takes not only average reward, but the variance of
each distribution observed so far into account.
The usage is to replace the confidence bound
\emph{$\sqrt{2\ln{n}/{n_j}}$} with another term composed of variance.
Thus the modified formulation can be expressed as:
\[\bar{x_j} + \sqrt{\frac{\ln{n}}{n_j} \min(\frac{1}{4},V_j(n_j))},\]
where $V_j(t) =\sigma_j^2 + \sqrt{\frac{2\ln n}{n_j}}$.

In conclusion, MAB problem is a template to describe the situation when
facing a uncertain problem with multiple distributions, and we would
like to evolve some strategy in order to minimize the degree of
regretfulness.
Among the solutions to MAB problems, UCB provides an intuitive solution
to illustrate the trade-off between exploration and exploitation.
Each round the recorded $\bar{X_j},\sigma_j,n_j$ contribute to the
measurement indicating which lever to pull.
According to the formulation, we can roughly tell that the choosing
strategy focuses on selecting a better $X_j$ or a less emphasized arm
with little $n_j$.
The 2-layer CMA-ES proposed in the previous section lacks the capability
of elitism.
In next section, we proposed an approach to model the problem in the
form of MAB problem.
Once we successfully mapped the problem, \emph{UCB1-tuned} is then
adopted to conquer the trade-off between exploration and exploitation
between groups.

\subsection{Methodology}

Recall that our approach is about discretization.
However, different from rECGA with SoD, this work aims to remain
possible division so that it can extract information between distributions
to reach the cliff.
On the other hand, rECGA with SoD tries to figure out the region where
global optimum exactly is.
As a result, we have to maintain a number of groups simultaneously where
the number is defined with respect to the population size according to
rule of thumb in \emph{k-means} clustering.

To transfer our hypothesis into a MAB problem, we can first regard $k$
groups as bandits.
A pull action is then an execution of CMA-ES with a specific number of
generation, and replacement occurs if the generated solutions is better
than the worst in the original group.
The representative solutions for groups and the method to evolve a
better solution according to selected candidates are the same to the
originally proposed 2-layer CMA-ES.
Under this assumption, we almost finished the transformation from the
assumption that better groups should be concentrated on to a MAB problem
introduced in the previous section.

However, the method meets a problem.
According to the proposed transformation, if the generation is given a
large number, as a result of the quick-convergence ability of CMA-ES,
the group may converge in the early stage.
Under this circumstance, we benefit nothing from the optimal pulling
sequence obtained by UCB because this does identical effort to the
original 2-layer CMA-ES since both use local optima as representative
solutions of groups, and thus the priority makes no sense.  

To overcome the difficulty and take advantage from the concept of
evolving groups, we can first adjust the number of generation to a
proper one in order to avoid a premature convergence.
Besides, the problem that priority makes no sense in the late stage is
not solved.
Since our system tried hard to maintain the diversity and the
transformation prepared above is insufficient to extract more
information than a pure 2-layer CMA-ES, we have to add some extra
modification.
The idea to overcome the problem is to design a policy that drop off
some groups that seem to contribute little or nothing to the overall
system.
After abandoning a group, we regard the solution generated by the $k-1$
representative solutions as a generator to form a new group by uniformly
generating solutions nearby.
By iteratively checking and replacing useless groups, we can keep
diversity along with taking advantage from the information evolved from
groups. 

In our modified system, the described \emph{UCB1-tuned} is integrated as
the policy for generating optimal pulling sequence.
However, the original form of \emph{UCB1-tuned} aims to find the maximal
$\bar{X_j}$ at each round.
On the other hand, our goal is to minimize the function value, so we
have to do some change to deal with the different target.
Since minimizing $\bar{X_j}$ is identical with maximizing $-\bar{X_j}$,
our demanding form should be rewritten as $ -\bar{X_j}
+\sqrt{\frac{\ln{n}}{n_j} \min(\frac{1}{4},V_j(n_j))}$.		

At last, we are performing the measurement of criteria to delete group.
The criteria are based on the idea that a group have to be abandon if
(1)the group has reached the local optimum of the region or (2)the group
is ignored for a long time.
The first criterion is easy to detect by a constant space requirement.
The second, however, is not theoretically defined according to an
informal assumption that the number of generation a group being ignored
should not exceed the number of solutions it consists of.




\begin{algorithm}[H] \caption{MAB-based CMA-ES} \KwIn{$n$,$t$ as a
	proper generation for each pulling action}
	\KwOut{best solution ever evolved} Uniformly sampled population of
	size $n$\; $k = \sqrt{\frac{n}{2}}$\; Integrating \emph{k-means}
	with \emph{Frogy-method} to cluster the $n$ individuals into $k$
	groups as known as bandits\; $C\leftarrow$ array with size $k$\;
	\For{i = 1 to k}            { pull($i$)\; $C_i\leftarrow$ best
	solution\; } \While{not terminated} { \For{i = 1 to k} {
			calculateUCB($i$) \tcp*[r]{calculate modified UCB1-tuned as
			illustrated above} record the index with max value in $M$\;

		} pull($M$)\;				$C_M \leftarrow$ best solution in
	group$_M$\; update()\; }
\end{algorithm}	

Comparing to the original 2-layer CMA-ES, this algorithm is featured in the
updating procedure.  From the flow of algorithm we can tell that an iteration
for the while loop is one round we refer to in MAB problem.  Each time a lever
is selected and pulled, the overall system faces a update. The
update checks in a random order if
any group meets any deletion criterion.  Assume a group $G$ is decided to be
destroyed,  we first generate a new solution $I$ as  a new group according to the local
optima without optimum of $G$, and we uniformly sample around $I$ to fill up the newly generated
group.  After a new group finishes constructing, the to-be-deleted group is then
replaced by the new group. The deletion occurs at most once a round,
and does nothing if no group meet deleting criteria. The update
procedure is outlined as following:

\begin{algorithm}
  Initial $P\leftarrow $ a permutation array from $1$ to $k$\; 
  ToBeDeleted = $0$\;
	\For{i = 1 to k} 
  {
    \If{deleting criterion 1 is met in group$_{P_i}$} 
    {
	    ToBeDeleted = $P_i$\; 
    }  

   }
   \If{ToBeDeleted = 0} 
   {
     \For{i = 1 to k} 
      {
				\If{deleting criterion 2 is met in group$_{P_i}$}
        {
	        ToBeDeleted = $P_i$\;
        } 
      } 
    } 
    \If{ToBeDeleted = 0} 
    { 
      return\; 
    } 
    \Else
  	{ 
      $s\leftarrow$ ToBeDeleted\; 
      generate a new solution as a new group	denoted as $group^\star$ according to $C_1, C_2,\ldots, C_{s_{i-1}}, C_{s_{i+1}},\ldots, C_k$ \; 
      \For{i = 1 to	$\|group_{s}\|-1$} 
      { pull($group^\star$) without replacing worst\; }
	replace $group_{s}$ with $group^\star$\; return\; 
} 
\caption{update}
\end{algorithm}

For further analysis, a newly generated group is with a small
number of times $n_j$ it has been pulled, thus the system tends to
pull this since smaller $n_j$ contributes more in the
bound value. As a result, if the implicit information is correctly
extracted, the evolution path of the 2-nd layer CMA-ES should be
driven to a better direction.  

In summary, this chapter provides two kinds of implementation for
the hypothesis given in the previous chapter. We adopt a clustering
method in order to maintain diversity of the population and further
investigate if there is a strategy to evolve not only in individuals
but in groups. The first implementation only examines the ability to
maintain a diversity if we apply \emph{k-means} clustering method. A
test to evolve the local optima of groups is given in order to
distinguish the existence of information hided among local optima.
In the second approach, we further investigate if information is
hided among the distributions. As a result, we control the speed of
the convergence and integrate UCB as an index to select a group to
evolve each round. The test case and experiment result are
introduced in the next chapter.
