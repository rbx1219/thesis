\chapter{Conclusion} \label{ch:conclusion}

In this work, a discretization framework for enhancing CMA-ES in a
rugged environment is proposed.
Different from traditional discretization approaches, such as FHH,
FWH, execute with fixed settings to address the discretization,
our work defines a self-adaptive method to achieve a soft
discretization.
The motivation is to deal with the difficulty that CMA-ES being an
outstanding local optimizer is easily trapped in non-convex, rugged
problems.
We are first inspired from the method commonly applied in evolutionary
algorithms which keeps diversity by maintaining a larger population.
Thus a modification on CMA-ES is needed since the original design of
CMA-ES is affected little by a large population size. 
The proposed MAB-based CMA-ES is an implementation features in splitting
large population into groups and evolving the implicit information among
groups.
The results in the test problems show that our approach is stabler even
in rugged fitness landscape comparing to rECGA, which is addressed by a
well-known improved discretization method, SoD.
On the other hand, we can obtain high resolution results once the
position of the global optimum is roughly located.
The comparison of the two algorithms is given in the previous chapter.
In the aspect of exploration, both do not dominate the other since they
solve some function while the other one is not able to reach the valley
during the given number of function evaluations.
To be brief, considering the best error values in the 25 runs of 25
problems, our algorithm wins in 14 problems and loses in 4.
Furthermore, considering the median error values, our algorithm wins in
18 problems while loses in 4.

In conclusion,this describes that our work properly performed in the
trade-off between exploration and exploitation.
By maintaining a huge population along with proper selection strategy,
we can further enhance the famous optimizer `CMA-ES' to gain more
diversity, which it originally lacks of. 
An implicit information exists between the distributions so that we can
obtain stabler results elegantly.

Although the design of our approach is still in early stage since the
deletion criteria, the number of generation for each pulling, etc.
is roughly demonstrated, the result achieved in this work presents a
possible way for developing soft discretization.
The improvement of designing more robust strategies for deleting
criteria and tuning parameters is expected.

